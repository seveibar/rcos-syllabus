\title{RCOS Syllabus Spring 2015}
\author{
    Rensselaer Center for Open Source
    \\
    UNOFFICIAL
}
\date{Amended \today}

\documentclass[12pt]{article}

\begin{document}
    \maketitle

    \tableofcontents

    \newpage

    \section{Overview}

    % Colin Rice's Introduction to RCOS
    % https://www.youtube.com/watch?v=s_O5B8WoFao

    Rensselaer Center for Open Source (RCOS, pronounced however you like) is class, research program and community. Students are given the opportunity to work on whatever project they like (provided it loosely adheres to the mission statement) while getting community feedback and support. While there are many experienced developers in RCOS, there is always a place for inexperienced but excited new developers.
    \subsection{Mission Statement}
    To provide a creative, intellectual and entrepreneurial outlet for students to use the latest open-source software platforms to develop applications that solve societal problems.
    \section{Learning Objectives}

    RCOS projects vary widely, employing many different technologies to solve a wide variety of problems.As a result, there's no way to really measure or even describe what you can or will learn from RCOS. Even so, there are some things you're guranteed to learn as an active RCOS member.

    \begin{itemize}
        \item Skills working or leading a small group of developers, delegating tasks, assessing workload, constraints and deadlines.
        \item Skills teaching or assisting other RCOS members
        \item Research techniques and ability to pick up new technologies quickly
        \item Presentation and projection skills
        \item Time management and self-motivating skills
    \end{itemize}

    \section{Contact Information}

    % TODO moorthy, goldschmidt, lead mentors, (mentors?)

    \section{Projects}

    \subsection{Acceptable Projects}

    % TODO

    \subsection{Team Responsibilities}

    % TODO

    \section{Organization}

    \subsection{Small Groups}

    Small groups are created at the beginning of the semester. Every RCOS member is in a small group with 2 to 3 mentors. Within each small group are project groups. A project group is limited in size to 4 people unless special permission is given.

    \subsection{Meetings}

    The organization of RCOS changes often, for this semester we've chosen to have a small group/large group meeting structure.

    There are 2 meetings a week, one on Tuesday and one on Friday. Attendance is taken at each meeting. The Tuesday meeting will be for small groups and the Friday meeting for everyone in RCOS. You're meeting place on Tuesdays will be given when your small group is assigned.

    \subsection{Presentations}

    Small group meetings are centered around discussion and technical help. Large group meetings are centered around presentations and general feedback.

    % TODO

    \section{Student Responsibilites}

    All students in RCOS are expected to remain active in the community, sharing with others their progress and problems as well as helping others.

    \begin{itemize}
        \item Share and accept feedback
        \item Attend meetings
        \item Help others with their projects
        \item Learn something new
    \end{itemize}

    \section{Getting Help}

    RCOS is centered around collaboration and learning is a group effort. If facing difficulties or stuck one should reach out for help from the community. If help is needed reach out in small group meetings, on Slack, through email or talk with a mentor. Chances are some one in RCOS may be able to help you or give help in the right direction.
    
    \section{Project Lead Responsibilites}

    % TODO

    \section{Mentor Responsibilites}

    Mentors are exemplary RCOS students with experience in open source and software development and want to help share their experience with the community. 

    \begin{itemize}
        \item Ensure continued progress on projects
        \item Provide guidance and help 
        \item Lead Small Group meetings and take attendance
        \item Work on a RCOS project
    \end{itemize}
    % TODO

    \section{Grading}

    % TODO


\end{document}
