\title{RCOS Syllabus Spring 2015}
\author{
    Rensselaer Center for Open Source
    \\
    UNOFFICIAL
}
\date{Amended \today}

\documentclass[12pt]{article}

\begin{document}
    \maketitle

    \tableofcontents

    \newpage

    \section{Overview}

    % Colin Rice's Introduction to RCOS
    % https://www.youtube.com/watch?v=s_O5B8WoFao

    Rensselaer Center for Open Source (RCOS, pronounced however you like) is class, research program and community. Students are given the opportunity to work on whatever project they like (provided it loosely adheres to the mission statement) while getting community feedback and support. While there are many experienced developers in RCOS, there is always a place for inexperienced but excited new developers.
    \subsection{Mission Statement}
    To provide a creative, intellectual and entrepreneurial outlet for students to use the latest open-source software platforms to develop applications that solve societal problems.
    \section{Learning Objectives}

    RCOS projects vary widely, employing many different technologies to solve a wide variety of problems.As a result, there's no way to really measure or even describe what you can or will learn from RCOS. Even so, there are some things you're guranteed to learn as an active RCOS member.

    \begin{itemize}
        \item Skills working or leading a small group of developers, delegating tasks, assessing workload, constraints and deadlines.
        \item Skills teaching or assisting other RCOS members
        \item Research techniques and ability to pick up new technologies quickly
        \item Presentation and projection skills
        \item Time management and self-motivating skills
    \end{itemize}

    \section{Contact Information}

    % TODO moorthy, goldschmidt, lead mentors, (mentors?)

    \section{Projects}

    The majority of learning in RCOS takes place within a student's project. This is where a student's can experiment with new technologies, build useful applications and collaborate with others.

    \subsection{Acceptable Projects}

    Projects can generally be about anything, as long as they loosely follow the RCOS mission statement. The mentors and faculty advisors will ultimately make the decision as to whether or not you can pursue your project. Usually if a project is not accepted, a suggestion is made that will still allow you to pursue a similar interest e.g. making a game engine that others can use instead of building a game.

    Hardware projects are also acceptable with the following restrictions.

    \begin{itemize}
        \item There must be some software aspect to your project (i.e. there must be code to open-source)
        \item You must adequately document the compoents/hardware you use such that others can learn from or use your project
        \item RCOS cannot pay for the components, you can pay for the components yourself or seek funding from the Embedded Hardware Club.
    \end{itemize}

    \subsection{Team Responsibilities}

    You don't have to work with a team in RCOS, but it is often beneficial to do so. As a team member your expected to do the following...

    \begin{itemize}
        \item Communicate with your team members and help each other when needed
        \item Share code frequently and review each others code
        \item Delegate work so that every team member knows what they should be working on
        \item Present together
    \end{itemize}


    \section{Organization}

    \subsection{Small Groups}

    Small groups are created at the beginning of the semester. Every RCOS member is in a small group with 2 to 3 mentors. Within each small group are project groups. A project group is limited in size to 4 people unless special permission is given.

    Although groups are selected by mentors, a student can opt to join any group or change mentors at any point in the semester. If a student is new to RCOS, they have the option to join a group with other new students.

    \subsection{Meetings}

    The organization of RCOS changes often, for this semester we've chosen to have a small group/large group meeting structure.

    There are 2 meetings a week, one on Tuesday and one on Friday. Attendance is taken at each meeting. The Tuesday meeting will be for small groups and the Friday meeting for everyone in RCOS. You're meeting place on Tuesdays will be given when your small group is assigned.

    \subsection{Presentations}

    Small group meetings are centered around discussion and technical help. Every group is expected to make presentations in their small groups detailing progress, struggles and milestones. Large group meetings are centered around presentations detailing releases, large progress updates, guest speakers and general feedback.

    \subsection{Slack}

    Slack is a communication platform similar to IRC (internet relay chat). This year we'll be testing to see if Slack is an effective medium for inter-project communication and getting help.

    Slack can be accessed from the web or mobile devices, and it's expected that every student check their project's Slack at least once a week. Slack can be accessed from \textbf{rcos.slack.com}

    \section{Student Responsibilites}

    All students in RCOS are expected to remain active in the community, sharing with others their progress and problems as well as helping others.

    \begin{itemize}
        \item Share and accept feedback
        \item Attend meetings
        \item Help others with their projects
        \item Learn something new
    \end{itemize}

    \section{Getting Help}

    RCOS is centered around collaboration and learning is a group effort. If facing difficulties or stuck one should reach out for help from the community. If help is needed reach out in small group meetings, on Slack, through email or talk with a mentor. Chances are some one in RCOS may be able to help you or give help in the right direction.

    \section{Project Lead Responsibilites}

    \begin{itemize}
        \item Ensure team progress on project
        \item Report progress and/or problems to mentors
        \item Make sure team members are contributing
        \item Ensure blog and Github repo is kept up to date
    \end{itemize}
    % TODO

    \section{Mentor Responsibilites}

    Mentors are exemplary RCOS students with experience in open source and software development and want to help share their experience with the community.

    \begin{itemize}
        \item Ensure continued progress on projects
        \item Provide guidance and help
        \item Lead Small Group meetings and take attendance
        \item Work on a RCOS project
    \end{itemize}
    % TODO

    \section{Grading}

    % TODO


\end{document}
